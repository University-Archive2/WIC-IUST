%% Last Update: 28 Jul 2019
%

%% ==========================================================================
% تنظیمات بسته geometry که مربوط به حاشیه صفحه می‌شود. 
\makeatletter
\ifthenelse{\equal{\@docStyle}{Report}}{\geometry{a4paper,top=2.4cm, bottom=2.6cm, left=2cm, right=2.1cm}}{
  \ifthenelse{\equal{\@docStyle}{Pres}}  {\geometry{a4paper,landscape,top=1.8cm,bottom=.7cm,left=1cm,right=1cm,includefoot,verbose,nohead,footskip=.5cm,bmargin=8mm}}{
    \geometry{a4paper,top=2.4cm, bottom=2.6cm, left=2cm, right=2.1cm} %Default
  }
}
\makeatother
% در Latex انواع مختلفی از اندازه‌ها برای ابعاد کاغذ وجود دارد. گرچه شما می‌توانید به دلخواه ابعاد مختلفی به کاغذ بدهید. 
%a0paper, a1paper, a2paper, a3paper, a4paper, a5paper, a6paper,b0paper, b1paper, b2paper, b3paper, b4paper, b5paper, b6paper, c0paper, c1paper, c2paper, c3paper, c4paper, c5paper, c6paper, b0j, b1j, b2j, b3j, b4j, b5j, b6j, ansiapaper, ansibpaper, ansicpaper, ansidpaper, ansiepaper, letterpaper, executivepaper, legalpaper


%% ==========================================================================
% تنظیمات listing مربوط به وارد کردن کد در متن
%% Date: 24 Jul 2019
%

%%% =========================================================================================================================
%%==================== تنظیمات listing
%%  در این قسمت تمام ابزارهای مورد نیاز در نوشتن برنامه ها اورده شده  است. با استفاده از این ابزارهای می‌توان برنامه های مورد نیاز را در مستند جای داد.
%% مطالب بیشتر در مورد وارد کردن کد در متن را در صفحه زیر مطالعه کنید.
%%%http://www.parsilatex.com/mediawiki/index.php?title=%D8%B1%D8%A7%D9%87%D9%86%D9%85%D8%A7%DB%8C_%D9%88%D8%A7%D8%B1%D8%AF_%DA%A9%D8%B1%D8%AF%D9%86_%DA%A9%D8%AF_%D8%AF%D8%B1_%D9%85%D8%AA%D9%86

\lstset{% general command to set parameter(s)
	% زبان برنامه نویسی که به طور پیش فرض انتخاب می شود.
	%language=Java,
	% رنگ پیش فرض برای پیش زمینه
	%backgroundcolor=\color{gray!10},
	%% میزان طول محیط listings را مشخص می کند، به صورت پیش فرض \textwidth است. 
	%linewidth=\textwidth ,
	%% نوع قالب دور محیط listings را تعیین می کند. 
	frameround=fttt,
	frame=single,
	aboveskip=5mm,
	belowskip=4mm,
	%% is selected at the beginning of each listing. You could use \footnotesize,
	%% \small, \itshape, \ttfamily, or something like that. The last token of
	%% basic style must not read any following characters.
	basicstyle=\setLTR\ttfamily, % print whole listing small
	%%   با این دستور استایل keyword ها را مشخص می کنیم. مثلا در این حالت گفته ایم که keyword ها را با رنگ آبی مشخص کند، و آن ها را bold‌کند. دقت کنید که keyword های زبان‌هایی که این بسته پشتیبانی می‌کند، 
	%% در این بسته تعریف شده است. مثلا در JAVA کلمه main به صورت پیش فرض تعریف شده است و در صورت وجود آن در کد شما آن را Latex آبی رنگ می‌کند. 
	keywordstyle=\color{blue}\bfseries,
	% underlined bold black keywords
	%identifierstyle=, % nothing happens
	%framexleftmargin=5mm, frame=shadowbox, rulesepcolor=\color{red}
	%% استایل String را در متن مشخص می کند. مثلا در این جا گفته شده است که رشته ها را با رنگ قرمز و به صورت ایتالیک نمایش بده.
	stringstyle=\ttfamily\color{red}, % typewriter type for strings
	%% نحوه استایل comment را مشخص می کند. دقت کنید که رنگ انتخاب شده نوعی رنگ سبز است، برای این که این رنگ شناخته شود می بایست دو بسته color و xcolor به صورتی که فراخوانی شده است، فراخوانی شود. 
	commentstyle=\color{Green},
	lineskip = 1mm,
	%% سه دستور بعدی نحوه نمایش شماره خطوط را مشخص می کند. 
	numberstyle=\footnotesize, 
	%% تعیین فاصله بین شماره خطوط و محیط listings
	numbersep=10pt,
	%% محل قرارگیری شماره خطوط
	numbers=left,
	xleftmargin=1pt,aboveskip=7mm,belowskip=6mm,
	%% تعیین محل قرارگیری caption محیط. بطور پیش فرض در بالای محیط است که به پایین محیط تغییر داده شده است. 
	%captionpos=b, 
	%% توسط breakline می توانید خاصیت شکسته شدن خطوط بلند را در محیط listings فعال و یا غیرفعال کنید.
	%% activates or deactivates automatic line breaking of long lines.
	breaklines=true,
	%% باعث می شود که فاصله های بین رشته های نمایان شود.
	%% lets blank spaces in strings appear  or as blank spaces
	%showstringspaces=true
	captiondirection=RTL
}%

\lstdefinestyle{customc}{
	belowcaptionskip=1\baselineskip,
	breaklines=true,
	frame=L,
	backgroundcolor=\color{white}
	xleftmargin=\parindent,
	language=C,
	showstringspaces=false,
	basicstyle=\setLTR\small\ttfamily,
	keywordstyle=\bfseries\color{green!40!black},
	commentstyle=\itshape\color{purple!40!black},
	stringstyle=\color{orange},
	numbers=none
}

\renewcommand{\lstlistingname}{\rl{کد}}

% البته شما می توانید این موارد پیش فرض را به ازای هر کد تغییر دهید. به عنوان مثال، ما یک کد در پوشه Code در شاخه فعلی قرار دادیم، می خواهیم آن را وارد متن کنیم، کافی است که خطوط زیر را در محل مناسبی که می خواهیم کد را قرار دهیم وارد کنیم. در این مثال یک فایل کد JAVA به نام myCode.java را می خواهیم وارد کنیم. 
%\begin{latin}
%\lstinputlisting[breaklines=true,numbers=left,language=Java, basicstyle=\ttfamily, numberstyle=\footnotesize, numbersep=10pt, captionpos=b, frame=single, breakatwhitespace=false]{Code/myCode.java}
%\end{latin}



%%% =========================================================================================================================
%تعریف و تکمیل Syntaxhighlight برای برخی از زبان‌های برنامه‌نویسی. 

%% ASN.1
\lstdefinelanguage[]{asn.1}%
{keywords=%
	{DEFINITIONS,SEQUENCE,OCTET,INTEGER,BEGIN,END,OPTIONAL,STRING,ENUMERATED,CHOICE,BOOLEAN},
	sensitive=true,
	morestring=[b]",
	morestring=[s]{>}{<},
	morecomment=[l]{--},
	morecomment=[s]{<?}{?>},
	stringstyle=\color{black},
	identifierstyle=\color{Blue},
	keywordstyle=\color{cyan},
	morekeywords={xmlns,version,type}% list your attributes here
	emph=[2]%
	{%
		DEFAULT,NULL,
	},
	emphstyle=[2]{\color{Red}},
	%
	emph=[3]% Variable Types
	{% 
		SIZE,
	},
	emphstyle=[3]{\color{Plum}},
}[keywords]%


%% XML
\lstdefinelanguage{XML}
{
  morestring=[b]",
  morestring=[s]{>}{<},
  morecomment=[s]{<?}{?>},
  stringstyle=\color{black},
  identifierstyle=\color{Blue},
  keywordstyle=\color{cyan},
  columns=fullflexible,
  showstringspaces=false,
  morekeywords={xmlns,version,type}% list your attributes here
}%


%% JSON
\colorlet{numb}{magenta!60!black}
\lstdefinelanguage{JSON}{
    stepnumber=1,
    numbersep=8pt,
    showstringspaces=false,
    breaklines=true,
    string=[s]{"}{"},
    comment=[l]{:\ "},
    morecomment=[l]{:"},
literate=
        *{0}{{{\color{numb}0}}}{1}
         {1}{{{\color{numb}1}}}{1}
         {2}{{{\color{numb}2}}}{1}
         {3}{{{\color{numb}3}}}{1}
         {4}{{{\color{numb}4}}}{1}
         {5}{{{\color{numb}5}}}{1}
         {6}{{{\color{numb}6}}}{1}
         {7}{{{\color{numb}7}}}{1}
         {8}{{{\color{numb}8}}}{1}
         {9}{{{\color{numb}9}}}{1}
}%


%% INI
\lstdefinelanguage{INI}
{
    morecomment=[s][\color{Orchid}\bfseries]{[}{]},
    morecomment=[l]{\#},
    morecomment=[l]{;},
    commentstyle=\color{gray}\ttfamily,
    morekeywords={},
    otherkeywords={=,:},
    keywordstyle={\color{Green}\bfseries}
}%


%% QT
\lstdefinelanguage{QT}
{
    language=C++,
    keywordstyle={\color{blue}\bfseries},
    keywordstyle=[2]{\color{Plum}\bfseries},
    keywordstyle=[3]\color{Orange},
    keywords=[2]{uint64_t,uint32_t,uint16_t,uint8_t,QByteArray,QObject,QString,QTimer,QSqlDatabase,QSqlQuery,QProcess,QSettings,QFileInfo,QJsonDocument,QStateMachine,QTextCodec,QJsonObject,QRegularExpression,QDomDocument,QXmlStreamReader,QRegExp,QHostAddress,QTextCodec,QMessageBox,QFile,QMainWindow,QDebug,QTime,QThreadPool},
    keywords=[3]{nullptr}
}%




%% ===========================================================================
%تنظیمات hyperref

% برای وارد کردن کلمه (بخش) در فهرست مطالب بسته hyperref برای حالت فارسی یک مشکل دارد. بدین منظور این 
% مشکل را به صورت دستی حل شده است. برای این که رنگ keywordstyle که تعیین کننده رنگ کل قسمت فهرست مطالب
% نیز هست یکسان در آید یک پارامتر رنگ برای keywordstyle این جا تعریف می‌کنیم، و سپس از آن هم در تنظمیات hypperref 
% و هم در اون کدهایی که به صورت دستی وارد شده است، استفاده می‌شود.  مطالب بیشتر در مورد این بسته را در سایت زیر مطالعه کنید.
% http://en.wikibooks.org/wiki/LaTeX/Hyperlinks

% در این قسمت تنظیمات بسته hyperref را قرار می دهیم.
% این تنظیمات شامل موارد زیر است:
\hypersetup{
	pdfmenubar=false,			% show or hide Acrobat’s menu
	pdftoolbar=true,			%show or hide Acrobat’s toolbar
%% موقعی که فایل پی دی اف خروجی را باز می کنید صفحه به صورت عریض و بزرگ باز می شود.
	pdfstartview=FitH, 
	%% مواردی که برای فعال سازی این که شماره اشکال را به صورت ارجاعی نشان دهد
	%hyperfigures=true,
	%% به جای استفاده از مربع قرمز دور موارد ارجاعی از لینک های رنگی استفاده کند.
	colorlinks=true, 
	%% رنگ برخی از لینک ها در زیر تعریف شده است. 
	linkcolor=blue, 
	anchorcolor=green, 
	citecolor=magenta, 
	urlcolor=cyan, 
	filecolor=magenta, 
	bookmarkstype=toc,
	unicode = true			%allows to use characters of non-Latin based languages in Acrobat’s bookmarks
	%bookmarksopen = true,
	%bookmarksopenlevel = 1
	%%% اگر این option را true‌ بکنیم، آن‌گاه در کنار bookmark شماره فصل و بخش و زیربخش نیز می آید. مثلا می‌نویسد: ۱.۲ طراحی شبکه
	%bookmarksnumbered = true,
	%hidelinks			%hide links (removing color and border)
} % M


%%% گاهی اوقات مطلبی را در یک بخش، فصل و ... می‌نویسیم، وقتی این مطلب را در متن ارجاع می دهیم، مثلا می نویسیم بخش ‎\ref{seclabel}‎، اما ممکن است به هر علتی مطلب ما از بخش به فصل و یا به یک زیربخش تنزل پیدا کند.بدین‌منظور به جای دستور ref از autoref استفاده می‌کنیم. در این صورت خود latex می‌فهمد که این مطالب در ‎چه قسمتی است و خودش کلمه فصل، بخش و یا ... را قبل از شماره آن اضافه می‌کند. دستورات زیر می‌گوید در هر یک  از سطوح متن چه کلمه‌ای‌ قرار گیرد. لازم به ذکر است که برای این کار به بسته hyperref نیاز است. 
\def\partname{پاره}
\renewcommand{\partautorefname}{پاره}
\def\sectionautorefname{بخش}
\def\subsectionautorefname{زیربخش}
\def\subsubsectionautorefname{زیربخش}
\def\equationautorefname{رابطه}
\def\figureautorefname{شکل}
\def\tableautorefname{جدول}
\def\lemmaautorefname{لم}


%% ====================================================================
%% تنظیمات tikz
%% لازم به ذکر است که کتابخانه های بسته tikz به صورت پیش فرض فعال نشده است، و کاربر به دلخواه خود اگر به مراتب به آن نیاز دارد باید آن را فعال کند. 
%\usetikzlibrary{mindmap,backgrounds,shadows,trees,arrows,shapes,positioning,shadings}
\usetikzlibrary{calc}


%% ====================================================================
%% تنظیمات algorithm و algorithmic
%\floatname{algorithm}{الگوریتم}


%% =====================================================================
%% تنظیمات graphicx
% برای اضافه کردن تصاویر به متن این امکان وجود دارد که تصاویر را در پوشه‌های متفاوت قرار داد. با این کار از زیاد شدن پرونده‌ها در مسیر مستند جلوگیری می شود. علاوه بر این دسته‌ای از تصاویر وجود دارد که بین همه مستندها مشترک است
% برای نمونه نماد پژوهشکده که بین همه مشترک است.  از این رو تعداد مسیر به عنوان مسیرهای پیش فرض برای جستجوی تصاویر تعیین شده است.
%\graphicspath{{../../Pic/}{./Pic/}}


%% =======================================================================
%%% تنظیمات مربوط به ایجاد watermarking
%
%%% زاویه متن Watermark
%\SetWatermarkAngle{45}
%%% اندازه watermark
%\SetWatermarkScale{1.5}
%
%\let\oldSetWatermarkText\SetWatermarkText
%%% اگر بخواهید watermark شما یک رنگ دیگر داشته باشد، این دو خط را فعال کنید و رنگ مورد نظر خود را انتخاب کنید
%%\definecolor{orange}{RGB}{229,252,219} 
%%\renewcommand{\SetWatermarkText}[1]{\oldSetWatermarkText{\textcolor{orange}{#1}}}
%
%\DeclareDocumentCommand{\SetWatermarkText}{m g}{
%	\oldSetWatermarkText{#1}
%	\IfValueTF{#2}{
%		\SetWatermarkLightness{#2}
%	}{%%
%		\SetWatermarkLightness{.94}
%	}%%
%}%


%% =========================================================================
% تنظیم فونت. 
%برای دانستن اطلاعات بیشتر در مورد تنظیم فونت به لینک زیر مراجعه کنید.
%http://www.parsilatex.com/mediawiki/index.php?title=%D9%81%D9%88%D9%86%D8%AA_%D8%AF%D8%B1_xepersian
% دقت شود که در این استایل از فونت های زیر استفاده می‌شود، بدین‌سان لازم است فونت هایی که فعال هستند (comment نیستند) توسط کاربر در سیستم عامل نصب شود.
%سعی شده است از فونت های استاندارد IR‌ که حاصل کار شورای عالی اطلاع رسانی هست، استفاده شود، کاربران می‌توانند فونت های یاد شده را از لینک زیر دانلود کنند.
% http://www.scict.ir/portal/Home/Default.aspx?CategoryID=b50ee619-ce53-4c25-bf53-b0f0332c1777

% تعریف یک دستور به عنوان فونت پیش فرض
\makeatletter
\newcommand{\defaultFont}{
% با دستور زیر می توانید فونتی مخصوص عبارات فارسی تعیین کنید:
	\ifthenelse{\equal{\@docStyle}{Report}} {\settextfont[Scale=1.25]{IRNazanin}}{
	\ifthenelse{\equal{\@docStyle}{Pres}}   {\settextfont[Scale=1.4]{IRNazanin}}{
	\ifthenelse{\equal{\@docStyle}{Letter}}{\settextfont[Scale=1.25]{IRNazanin}}{}}
	}
% شما با دستور زیر بعد از فراخوانی بسته xepersian می توانید فونت انگلیسی را تعیین کنید
% دقت کنید که عبارات انگلیسی شما باید در دستور \lr{} قرار گیرد تا xepersian بتواند بفمهد که این عبارات انگلیسی است
	\ifthenelse{\equal{\@docStyle}{Report}} {\setlatintextfont[Scale=1.1]{Linux Libertine}}{
	\ifthenelse{\equal{\@docStyle}{Pres}}   {\setlatintextfont[Scale=1.2]{Linux Libertine}}{
	\ifthenelse{\equal{\@docStyle}{Letter}}{\setlatintextfont[Scale=1.1]{Linux Libertine}}{}}
	}
% تعریف برای فونت اعداد و ارقام
	%\setdigitfont[Scale=1.1]{XB Zar}
} % M
\makeatother

\defaultFont

%%  با استفاده از این دستور می‌توان فونت و فارسی و یا انگلیسی بودن اعداد در فرمول‌ها را به حالت اولیه (یعنی پیش‌فرض لاتک) برگرداند.
%\DefaultMathsDigits

%% نحوه تغییر اندازه فونت عبارات ریاضی و فرمول‌ها. این کار توسط دستور زیر انجام می‌شود. 
%%\DeclareMathSizes{textsize}{mathsize}{scriptsize}{scriptscriptsize}
%% گزینه اول: این برای چه دسته فونتی است. پیش فرض استایل ما فونت 10pt است. 
%% گزینه دوم: اندازه فونت توابع و موجودات ریاضی درون متن.
%% گزینه سوم: برای اسکریپت ها، اندازه زیرنویس و بالانویس.
%% گزینه چهارم: برای زیرنویس زیرنویس.

%% در دستورات زیر ما برای سه حالت، اندازه‌های مورد نظر را تعریف کرده ایم. 
%%\DeclareMathSizes{10}{11}{9}{8}   % For size 10 text
%%\DeclareMathSizes{11}{12}{11}{10}   % For size 11 text
%%\DeclareMathSizes{12}{13}{12}{11}  % For size 12 text
