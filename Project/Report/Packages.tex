\usepackage{ptext}
\usepackage{graphicx} 
%%%Packages added for Math%%%%
\usepackage{optidef}
%% Packages for Table
\usepackage{multirow}
\usepackage[table,xcdraw]{}
%%
\usepackage{sidecap}%%

\usepackage{listings}

%%pachages for sub-figure
\usepackage{subcaption}
\usepackage{xepersian}
\settextfont[Scale=1.2]{XB Niloofar}
\setlatintextfont[Scale=1]{Times New Roman}

%% برای وارد کردن کد در برنامه
\usepackage{listings}
%%محیط های زیبای لتک
%\usepackage[usenames,dvipsnames]{color,}
%\usepackage{tikz,times}
%\usepackage{xparse}
%\usepackage{ifthen}
%\usepackage{pifont}

%زی‌پرشین (به انگلیسی: XePersian) یک بسته حروف‌چینی رایگان و متن‌باز برای نگارش مستندات پارسی/انگلیسی با زی‌لاتک است.
% در واقع، زی‌پرشین، کمک می‌کند تا به آسانی، مستندات را به پارسی، حروف‌چینی کرد. این بسته را وفا خلیقی نوشته است،
% و به طور منظم، آن را بروز‌رسانی کرده و باگ‌های آن را رفع می‌کند.
% نکته مهم این جا است که بسته Xepersian برای پشتیبانی از زبان فارسی آورده شده است، و 
% می بایست آخرین بسته ای باشد که شما وارد می کنید، دقت کنید: آخرین بسته

%%  با دستور زیر می توانید فونتی مخصوص عبارات فارسی تعیین کنید:

%% شما با دستور زیر بعد از فراخوانی بسته xepersian می توانید فونت انگلیسی را تعیین کنید
%% دقت کنید که عبارات انگلیسی شما باید در دستور \lr{} قرار گیرد تا xepersian بتواند بفمهد که این عبارات انگلیسی است
